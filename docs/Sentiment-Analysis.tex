% !TEX encoding = UTF-8 Unicode
\documentclass[a4paper]{article}

\usepackage{color}
\usepackage{url}
\usepackage[T2A]{fontenc} % enable Cyrillic fonts
\usepackage[utf8]{inputenc} % make weird characters work
\usepackage{graphicx}

\usepackage[english,serbian]{babel}
%\usepackage[english,serbianc]{babel} %ukljuciti babel sa ovim opcijama, umesto gornjim, ukoliko se koristi cirilica

\usepackage[unicode]{hyperref}
\hypersetup{colorlinks,citecolor=green,filecolor=green,linkcolor=blue,urlcolor=blue}

\usepackage{listings}

%\newtheorem{primer}{Пример}[section] %ćirilični primer
\newtheorem{primer}{Primer}[section]

\definecolor{mygreen}{rgb}{0,0.6,0}
\definecolor{mygray}{rgb}{0.5,0.5,0.5}
\definecolor{mymauve}{rgb}{0.58,0,0.82}

\lstset{ 
  backgroundcolor=\color{white},   % choose the background color; you must add \usepackage{color} or \usepackage{xcolor}; should come as last argument
  basicstyle=\scriptsize\ttfamily,        % the size of the fonts that are used for the code
  breakatwhitespace=false,         % sets if automatic breaks should only happen at whitespace
  breaklines=true,                 % sets automatic line breaking
  captionpos=b,                    % sets the caption-position to bottom
  commentstyle=\color{mygreen},    % comment style
  deletekeywords={...},            % if you want to delete keywords from the given language
  escapeinside={\%*}{*)},          % if you want to add LaTeX within your code
  extendedchars=true,              % lets you use non-ASCII characters; for 8-bits encodings only, does not work with UTF-8
  firstnumber=1000,                % start line enumeration with line 1000
  frame=single,	                   % adds a frame around the code
  keepspaces=true,                 % keeps spaces in text, useful for keeping indentation of code (possibly needs columns=flexible)
  keywordstyle=\color{blue},       % keyword style
  language=Python,                 % the language of the code
  morekeywords={*,...},            % if you want to add more keywords to the set
  numbers=left,                    % where to put the line-numbers; possible values are (none, left, right)
  numbersep=5pt,                   % how far the line-numbers are from the code
  numberstyle=\tiny\color{mygray}, % the style that is used for the line-numbers
  rulecolor=\color{black},         % if not set, the frame-color may be changed on line-breaks within not-black text (e.g. comments (green here))
  showspaces=false,                % show spaces everywhere adding particular underscores; it overrides 'showstringspaces'
  showstringspaces=false,          % underline spaces within strings only
  showtabs=false,                  % show tabs within strings adding particular underscores
  stepnumber=2,                    % the step between two line-numbers. If it's 1, each line will be numbered
  stringstyle=\color{mymauve},     % string literal style
  tabsize=2,	                   % sets default tabsize to 2 spaces
  title=\lstname                   % show the filename of files included with \lstinputlisting; also try caption instead of title
}

\begin{document}

\title{Analiza sentimenta u filmskim recenzijama\\ \small{Seminarski rad u okviru kursa\\Računarska inteligencija\\ Matematički fakultet}}

\author{Mihajlo Vićentijević, Marko Vićentijević}

%\date{9.~april 2015.}

\maketitle

\abstract{
Sažetak

\tableofcontents

\newpage

\section{Postavka problema}
Mišljenja, recenzije, preporuke, stavovi i ostale ljudske ekspresije predstavljaju vredan resurs jer odražavaju njihovo zadovoljstvo proizvodima, uslugama, političkim stavovima. Ovakve povratne informacije mogu se iskoristiti za pobošljavanje nekih elemenata sistema: kvalitet proizvoda, marketinški pristup, poslovna strategija, kreiranje političke kampanje, itd. U tom smislu ćemo prikupiti i analizirati filmske recenzije primenom mašinskog učenja na analizu sentimenta. Analiza sentimenta je popularna poddisciplina oblasti NLP-a i bavi se analizom polariteta. Cilj nam je da na osnovu izraženih mišljenja ili emocija klasifikujemo filmske recenzije u pozitivne ili negativne. Koristićemo skup podataka \textit{SerbMR}. \cite{batanovic2016reliable}


\section{Opis podataka}
Podaci obuhvataju 1682 filmskih recenzija (841 pozitivnih i 841 negativnih).
Recenzije su prikupljene sa sledećih veb lokacija:
\begin{itemize}
  \item 2kokice.com
  \item filmskerecenzije.com
  \item filmskihitovi.blogspot.com
  \item happynovisad.com
  \item kakavfilm.com
  \item mislitemojomglavom.blogspot.com
  \item popboks.com
  \item yc.rs
\end{itemize}
Obuhvaćene recenzije su nastale do 01.01.2015. i pretežno su napisane u ekavskom dijalektu.Recenzije koje su napisane ćirilicom su prevdene u latinicu.

\section{Metodologije i implementacija}

\section{Eksperimentalni rezultati}

\section{Zaključak}

\addcontentsline{toc}{section}{Literatura}
\appendix
\bibliography{Sentiment-Analysis} 
\bibliographystyle{plain}

\end{document}
